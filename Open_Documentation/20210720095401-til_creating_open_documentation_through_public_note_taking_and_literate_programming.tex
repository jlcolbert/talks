% Created 2021-11-01 Mon 12:39
% Intended LaTeX compiler: pdflatex
\documentclass[11pt]{article}
\usepackage[utf8]{inputenc}
\usepackage[T1]{fontenc}
\usepackage{graphicx}
\usepackage{longtable}
\usepackage{wrapfig}
\usepackage{rotating}
\usepackage[normalem]{ulem}
\usepackage{amsmath}
\usepackage{amssymb}
\usepackage{capt-of}
\usepackage{hyperref}
\author{Jay L. Colbert}
\date{\today}
\title{TIL:\\\medskip
\large Creating Open Documentation Through Public Note-taking and Literate Programming}
\hypersetup{
 pdfauthor={Jay L. Colbert},
 pdftitle={TIL:},
 pdfkeywords={},
 pdfsubject={},
 pdfcreator={Emacs 27.2 (Org mode 9.6)}, 
 pdflang={English}}
\begin{document}

\maketitle
\setcounter{tocdepth}{1}
\tableofcontents


\section*{Who am I?}
\label{sec:org0a73ec5}
\begin{itemize}
\item Jay L. Colbert
\item he/him pronouns
\item Metadata \& Discovery Strategy Librarian
\item University of New Hampshire
\end{itemize}

Bachelor's degree in\ldots{}

Literature

\section*{Learning Primo VE}
\label{sec:org0d47203}
In September 2019, I was hired as the University of New Hampshire's Metadata \& Discovery Strategy Librarian.

One of my primary position responsibilities is managing our Primo VE Discovery Layer, which UNH had implemented in July 2019.

\subsection*{Challenges}
\label{sec:org6ab7b5e}
\begin{itemize}
\item New to software
\item No formal tech background beyond metadata
\item Primo? Primo NUI? Primo VE?
\item Documentation isn't always clear
\item A lot of documentation and third-party information is years out of date!
\end{itemize}

The instructions for the Primo Development Environment \emph{do} contain some information for Primo VE.

But most of the third-party packages do not.

And many of the third-party packages also use the \texttt{-{}-browserify} option\ldots{}

And Ex Libris does not provide official documentation for using it.

\subsection*{Solutions}
\label{sec:orgbc45faf}
\begin{itemize}
\item Explore GitHub repositories
\begin{itemize}
\item Commits
\item Issues
\item Pull requests
\end{itemize}
\item Watch conference presentations
\item ``Mess around and find out''
\end{itemize}

\section*{The Importance of Documentation}
\label{sec:orge15a6dc}
\begin{enumerate}
\item A single source of truth saves time and energy
\item Documentation is essential to quality and process control
\item Documentation cuts down duplicative work
\item Documentation makes hiring and onboarding easier
\item Documentation increases collective knowledge
\end{enumerate}

With documentation, we are able to share our knowledge with others internally and externally.

Without documentation, knowledge is lost over time.
Consistency is lost.
Quality control is lost.

\subsection*{System-based}
\label{sec:orgee8839b}
``here's what the \emph{system does}''
\subsection*{Task-Based}
\label{sec:org18d0b23}
``here's how to do \emph{your} job \emph{using the system}''
\section*{Learning in Public}
\label{sec:orgc77f799}
\begin{quote}
“Learning in Public” is scary for many reasons – people can find and cling to outdated information and users are exposing their knowledge during a vulnerable time in the project (i.e. when they don’t yet have all the answers).

However, during this part of the process is when learning can be most valuable.

— via How Do Rocket Scientists Learn? (aka, knowledge management lessons learned at Goddard, NASA)
\end{quote}

\begin{itemize}
\item Track learning process
\item Public log
\item Continual feedback
\item Tap into network of community knowledge
\end{itemize}

\section*{Methods}
\label{sec:org9ce2302}
\subsection*{Beginner: Git Repository}
\label{sec:org5988de3}
Example: \texttt{https://github.com/jbranchaud/til}

\subsection*{Intermediate: Digital Garden}
\label{sec:orgc6b93fc}
\begin{quote}
The phrase ``digital garden'' is a metaphor for thinking about writing and creating that focuses less on the resulting ``showpiece'' and more on the process, care, and craft it takes to get there.

We gather and work together in community gardens to share the labor as well as the rewards of a collective effort.

It's a comparison that you can take very far. From ``planting seeds'' and ``pulling weeds'' to tending mutiple gardens that each serve an individual need or desired outcome.

Like with real gardens, our digital gardens are a constant ebb and flow towards entropy.

— My blog is a digital garden, not a blog, by Joel Hooks
\end{quote}

\subsubsection*{Tools}
\label{sec:orgeb44e28}
All tools are free and open source.
I have purposefully omitted tools which are not free (like Roam Research) and/or open source (like Obsidian)

\texttt{https://github.com/MaggieAppleton/digital-gardeners}

\begin{itemize}
\item Dendron
\item Foam
\item Git Repository
\item Logseq
\item Neuron
\item Org Roam
\item TiddlyWiki
\end{itemize}

\subsection*{Advanced: Literate Programming}
\label{sec:orga66a5e8}
\subsubsection*{Literate programming}
\label{sec:org05b44c5}
\begin{itemize}
\item is a style and paradigm of programming and documentation
\item emphasizes natural language and human logic
\item embeds code snippets within documentation
\item generates software from documentation instead of the converse
\item encourages reproducible research and open access
\end{itemize}

\subsubsection*{Tools}
\label{sec:org339b2fd}
\begin{itemize}
\item NoWEB
\item Literate
\item pyWeb
\item Emacs org-mode*
\item Codebraid
\item Jupyter Notebook
\end{itemize}

You can combine these methods and tools:

host your digital garden(s) (using whatever tool) in a Git repository (as well as a website if desired) and include any literate programming documents within.

\section*{Questions?}
\label{sec:org3587f19}
email: jay.colbert@unh.edu

GitHub: jlcolbert

twitter: \_WildeAtHeart
\end{document}
