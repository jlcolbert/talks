% Created 2021-11-01 Mon 13:55
% Intended LaTeX compiler: pdflatex
\documentclass[11pt]{article}
\usepackage[utf8]{inputenc}
\usepackage[T1]{fontenc}
\usepackage{graphicx}
\usepackage{longtable}
\usepackage{wrapfig}
\usepackage{rotating}
\usepackage[normalem]{ulem}
\usepackage{amsmath}
\usepackage{amssymb}
\usepackage{capt-of}
\usepackage{hyperref}
\author{Jay L. Colbert}
\date{\today}
\title{Finding LGBTQ+ Materials at the UNH Library and Beyond}
\hypersetup{
 pdfauthor={Jay L. Colbert},
 pdftitle={Finding LGBTQ+ Materials at the UNH Library and Beyond},
 pdfkeywords={},
 pdfsubject={},
 pdfcreator={Emacs 27.2 (Org mode 9.6)}, 
 pdflang={English}}
\begin{document}

\maketitle
\tableofcontents



\section*{Important Info}
\label{sec:org11db322}

Some of the language I use today may be considered out of date or offensive.

However, it is important to know their historical importance and/or how they are still used today by libraries.

\section*{Who am I?}
\label{sec:orga2452bf}

Jay L. Colbert

he/him pronouns

Homosaurus Editorial Board

Metadata \& Discovery Strategy Librarian

\subsection*{What does that mean?}
\label{sec:org3c2904c}

Metadata is ``data about data''.

A more helpful definition might be ``the sum total of what one can say at a given moment about any information object''.

You encounter metadata all the time:

\begin{itemize}
\item file names
\item the info on your license or ID
\item the titles of YouTube videos
\item and more!
\end{itemize}

\subsection*{So why are you telling me about metadata?}
\label{sec:orgb618add}

You also encounter metadata any time you search for something online or in the library.

The library catalog, Amazon, and Google take your search terms and run them against a vast index of metadata.
Then, they show you the items in their catalog that match your keywords.

Knowing even a little bit about this process can help you when searching for any information online.

\section*{Why can it be so hard to find LGBTQ materials?}
\label{sec:org623a7d5}

Lavender linguistics:

\begin{itemize}
\item LGBTQ slang
\item Polari
\item Historical changes
\item Culturally-specific
\end{itemize}

Much of this terminology is insular, meaning it is by various LGBTQ communities, for those communities, as a way to communicate safely without cisgender and/or heterosexual people understanding them.

This is crucial in times and places when queerness is criminalized or stigmatized.

\subsection*{Use in Libraries}
\label{sec:org3e7b197}

Unsurprisingly, this language is not always the language used by libraries to describe materials.

Most libraries in the United States assign Library of Congress Subject Headings to materials, but there are other vocabularies for different contexts.

\emph{To this day}, there is no Library of Congress Subject Heading for a queer identity or LGBTQ community.

Instead, we get \emph{Sexual minorities} and \emph{Gender minorities}.

This is the tip of the iceberg.

Remember that index of metadata?

If a record doesn't have your search terms, \emph{even if it's talking about what you're searching for}, it won't show up in your search results.

\section*{Search Strategies}
\label{sec:org063b1cb}

\subsection*{UNH Library}
\label{sec:org1f0a064}

Library Search Box and databases

How can you search?

\begin{itemize}
\item keywords
\item subject headings
\item author
\item and more
\end{itemize}

When you search, look at \emph{how} materials are described.

If you find a book or article that is \emph{exactly} what you need, use the language used to describe it when you search!

Different disciplines, authors, and contexts will use different language to describe same or similar topics.

Let's do a demonstration!

\subsection*{Beyond}
\label{sec:orga9bd75f}

Google doesn't really do Boolean anymore.

Its algorithms are biased and skewed towards sites with high engagement.

This tends to lead people towards more controversial or right-wing sites.

Google Scholar doesn't have a useful search interface, either.

Helpful for seeing citations, certain years, broad searches, and seeing what disciplines are talking about a topic.

Books \& Media Worldwide:

\begin{itemize}
\item doesn't include articles
\item refined search like the library search box
\item Interlibrary Loan
\end{itemize}

General Tips:

\begin{itemize}
\item mind read: use a variety of synonyms that mean the same thing, and try to guess what language a discipline might use
\item use the External Search options in the Library Search Box to repeat your search in Google Scholar and Books \& Media Worldwide
\end{itemize}


\section*{What can I do as a scholar?}
\label{sec:org9ee3db2}

Include queer terminology in:

\begin{itemize}
\item author-supplied keywords
\item titles
\item abstracts
\item descriptions
\end{itemize}

\section*{Resources}
\label{sec:orgb9aa760}

\begin{itemize}
\item www.digitaltransgenderarchive.net
\item gsso.research.cchmc.org/\#!/
\item www.glbthistory.org/digital-collections
\item histsex.org
\item homosaurus.org
\item www.loc.gov/collections/lgbtq-studies-web-archive/
\item queerdigital.com
\item www.netanelganin.com/projects/QueerLCSH/QueerLCSH.html
\item archive.qzap.org
\item Your subject librarians!
\end{itemize}

\section*{Questions?}
\label{sec:org02bd9e0}

jay.colbert@unh.edu
\end{document}
